%! Tex program = xelatex

\documentclass[a4paper]{article}

\usepackage[utf8]{inputenc}
\usepackage[T1]{fontenc}
\usepackage{textcomp}
\usepackage[english]{babel}

% figure support
\usepackage{import}
\usepackage{xifthen}
\usepackage{pdfpages}
\usepackage{transparent}

\usepackage{physics}
\usepackage{braket}
\usepackage{mathtools}

\usepackage{float}
\usepackage{listings}

\usepackage{ctex}
\usepackage{amsthm}
\usepackage{amsmath}
\usepackage{amssymb}
\usepackage{tikz}
\usepackage{graphicx}
\usepackage{wasysym}

\usepackage[utf8]{inputenc}
\usepackage{cite}
\usepackage{amsfonts}
\usepackage{algorithmic}
\usepackage{textcomp}
\usepackage{xcolor}
\usepackage{mhchem}
\usepackage{siunitx}
\usepackage{xcolor}
\usepackage{tikz}
\usepackage{indentfirst}
\usepackage{caption}

\usepackage{pythonhighlight}

\usepackage[siunitx, RPvoltages]{circuitikz}

% Default fixed font does not support bold face
\DeclareFixedFont{\ttb}{T1}{txtt}{bx}{n}{12} % for bold
\DeclareFixedFont{\ttm}{T1}{txtt}{m}{n}{12}  % for normal

% Custom colors
\usepackage{color}
\definecolor{deepblue}{rgb}{0,0,0.5}
\definecolor{deepred}{rgb}{0.6,0,0}
\definecolor{deepgreen}{rgb}{0,0.5,0}

% % Python style for highlighting
% \newcommand\pythonstyle{\lstset{
% language=Python,
% basicstyle=\ttm,
% morekeywords={self},              % Add keywords here
% keywordstyle=\ttb\color{deepblue},
% emph={MyClass,__init__},          % Custom highlighting
% emphstyle=\ttb\color{deepred},    % Custom highlighting style
% stringstyle=\color{deepgreen},
% frame=tb,                         % Any extra options here
% showstringspaces=false
% }}
% 
% 
% % Python environment
% \lstnewenvironment{python}[1][]
% {
% \pythonstyle
% \lstset{#1}
% }
% {}
% 
% % Python for external files
% \newcommand\pythonexternal[2][]{{
% \pythonstyle
% \lstinputlisting[#1]{#2}}}
% 
% % Python for inline
% \newcommand\pythoninline[1]{{\pythonstyle\lstinline!#1!}}

% \usepackage{xeCJK}
% \setCJKmainfont{STXihei}

\graphicspath{ {./images/} }

\usetikzlibrary{positioning}
\usetikzlibrary{tikzmark,calc,decorations.pathreplacing}

\theoremstyle{theorem}
\newtheorem{theorem}{Theorem}[section]

\theoremstyle{definition}
\newtheorem{definition}{Definition}[section]

\theoremstyle{lemma}
\newtheorem{lemma}[theorem]{Lemma}

\theoremstyle{corollary}
\newtheorem{corollary}{Corollary}[theorem]

\theoremstyle{remark}
\newtheorem*{remark}{Remark}

\theoremstyle{axiom}
\newtheorem{axiom}{Axiom}[section]

\newcommand{\D}{\mathrm{d}}

\title{}
\author{brianShan974}
\date{\today}

\begin{document} % \maketitle

\section*{Problem 1}

The \(a_V\) term increases binding energy since binding energy is proportional to the number of nucleons.
The \(a_s\) term decreases binding energy since the nucleons on the surface of the nucleus has no binding partners.
The \(a_C\) term decreases binding energy since protons repel each other.
The \(a_a\) term decreases binding energy since there are usually more neutrons than protons.
The effect of \(a_p\) term depends on \(N\) and \(Z\).
When \(N\) and \(Z\) are both even,
the spin of the nucleus is an integer thus obey Bose-Einstein statistics.
The wavefunction can then overlap and the nuleons are tighter bound,
and binding energy is maximised.
Similarly,
when \(N\) and \(Z\) are both odd,
binding energy is minimised.
When one of them is even and the other is odd,
the total effect is 0.

\section*{Problem 2}

Start by taking the derivative
\[
  \derivative{M}{Z} = m_p - m_n + 2a_cA^{-\frac{1}{3}}Z + 2a_a\frac{Z - \frac{A}{2}}{A}
.\]
By setting the derivative to \(0\) we get
\[
  (2a_cA^{-\frac{1}{3}} + 2a_aA^{-1})Z = m_n - m_p + a_a
.\]
Thus we have
\[
  Z = \frac{m_n - m_p + a_a}{2a_cA^{-\frac{1}{3}} + 2a_aA^{-1}} = 60.67
.\]
We then compare \(M(60, 145) = 134952.43001949298\si{MeV}\) and \(M(61, 145) = 134952.16351767798\si{MeV}\) to see that \(Z = 61\) is the most stable nucleus.

\section*{Problem 3}

Fusion is possible up to \ce{^26Fe} because it has the maximal value of binding energy per nucleon.
The product of fusion of \ce{^26Fe} with any smaller nuclei cannot have a nucleus that has a larger value of binding energy per nucleon.

\section*{Problem 4}

The ground state is given by
\begin{align*}
  N:& (1s_{\frac{1}{2}})^2(1p_{\frac{3}{2}})^3 \\
  Z:& (1s_{\frac{1}{2}})^2(1p_{\frac{3}{2}})^2
\end{align*}
If only neutrons are excited,
the mostly likely configurations are given by
\begin{align*}
  N:& (1s_{\frac{1}{2}})^2(1p_{\frac{3}{2}})^3(1p_{\frac{1}{2}})^1 \\
  Z:& (1s_{\frac{1}{2}})^2(1p_{\frac{3}{2}})^2
\end{align*}
and
\begin{align*}
  N:& (1s_{\frac{1}{2}})^1(1p_{\frac{3}{2}})^4 \\
  Z:& (1s_{\frac{1}{2}})^2(1p_{\frac{3}{2}})^2
\end{align*}

\section*{Problem 5}

For \ce{^40_18Ar}:
\begin{align*}
  N&:\, (1s_{\frac{1}{2}})^2(1p_{\frac{3}{2}})^4(1p_{\frac{1}{2}})^2(1d_{\frac{5}{2}})^6(1d_{\frac{3}{2}})^4(1f_{\frac{7}{2}})^2(2s_{\frac{1}{2}})^2 \\
  Z&:\, (1s_{\frac{1}{2}})^2(1p_{\frac{3}{2}})^4(1p_{\frac{1}{2}})^2(1d_{\frac{5}{2}})^6(1d_{\frac{3}{2}})^4(2s_{\frac{1}{2}})^2 \\
\end{align*}
\ce{^40_18Ar} has even numbers of protons and neutrons,
and all of them are paired.
As a result,
the spin and parity are \(J^P = 0^+\). \\
For \ce{^39_19K}:
\begin{align*}
  N&:\, (1s_{\frac{1}{2}})^2(1p_{\frac{3}{2}})^4(1p_{\frac{1}{2}})^2(1d_{\frac{5}{2}})^6(1d_{\frac{3}{2}})^4(2s_{\frac{1}{2}})^2 \\
  Z&:\, (1s_{\frac{1}{2}})^2(1p_{\frac{3}{2}})^4(1p_{\frac{1}{2}})^2(1d_{\frac{5}{2}})^6(1d_{\frac{3}{2}})^3(2s_{\frac{1}{2}})^2 \\
\end{align*}
In \ce{^40_19K}, all neutrons are paired.
There is one unpaired proton in \(1d_{\frac{3}{2}}\) level with \(l = 2\) and \(j = \frac{3}{2}\),
so \(J^P = \left(\frac{3}{2}\right)^+\). \\
For \ce{^40_20Ca}:
\begin{align*}
  N&:\, (1s_{\frac{1}{2}})^2(1p_{\frac{3}{2}})^4(1p_{\frac{1}{2}})^2(1d_{\frac{5}{2}})^6(1d_{\frac{3}{2}})^4(2s_{\frac{1}{2}})^2 \\
  Z&:\, (1s_{\frac{1}{2}})^2(1p_{\frac{3}{2}})^4(1p_{\frac{1}{2}})^2(1d_{\frac{5}{2}})^6(1d_{\frac{3}{2}})^4(2s_{\frac{1}{2}})^2 \\
\end{align*}
In \ce{^40_20Ca},
all levels are full-filled.
Therefore,
\(J^P = 0^+\).

\ce{^40_20Ca} is stable because it is doubly magic.

\section*{Problem 6}

We can first calculate \(j\) by \(16 = 2j + 1\),
then \(j = \frac{15}{2}\).
We then have \(l = j\pm \frac{1}{2}\) so that \(l\) is either \(7\) or \(8\).
Given that it has an odd parity,
\(l\) has to be \(7\).

\section*{Problem 7}

The half-lime is given by
\[
  T_{\frac{1}{2}} = 1600\si{y} = \SI{50457600000}{s}
.\]
The lifetime is then given by
\[
  \tau = \frac{T_{\frac{1}{2}}}{\ln 2} = \SI{72794929295.15897}{s}
.\]
The total number of \ce{^226_88Re} atoms, \(N\), can then be estimated as
\[
  N = \SI{0.37}{Bq} * \tau = 26934123839.208817
.\]
The amount of substance is then
\[
  n = \frac{N}{N_A} = 4.472516487510467\times 10^{14}\si{mol}
.\]
The total mass is
\begin{align*}
  m &= nM_{\ce{^226_88Re}} \\
    &= 4.472516487510467\times 10^{14}\si{mol} \times \SI{226.02541}{g/mol} \\
    &= 1.0109023728213132\times 10^{-11}\si{g}
\end{align*}
Finally, the percentage is given by
\[
  \text{percentage} = \frac{m}{\SI{1}{g}} \times 100\% = 1.01\times 10^{-9}\%
.\]

\section*{Problem 8}

\[
  \text{fission rate} = \frac{\SI{10}{kW}}{\SI{200}{MeV}} = 6.24\times 10^{16}\si{s^{-1}}
.\]

\end{document}
